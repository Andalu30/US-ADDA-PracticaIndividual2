\documentclass[a4paper,12pt]{article}
%\usepackage[utf8]{inputenc} %no necesario si se usa LuaLaTeX
\usepackage{amsmath}
\usepackage[spanish]{babel}
\usepackage{listings}
\usepackage{minted} %La hostia!
\usepackage[margin=25mm]{geometry} %Margenes un poco mejor
\usepackage{graphicx}
\graphicspath{{resources/}}
\usepackage{hyperref} %URL de github y gnu

\title{ADDA - Practica Individual 2\\ Problema 3}
\author{Juan Arteaga Carmona\\(juan.arteaga41567@gmail.com)\\2º Curso - TI-2}
\date{\today}

\begin{document}
\maketitle

\section{Completar la ficha de descripción del problema}
\section{Resolver el problema por PD, para ello:}
\subsection{Completar la ficha de descripcion de la solución mediante programacion dinámica.}
\subsection{Escriba un archivo denominado "alimentos.txt" con los datos del escenario de entrada de forma similar a como se ha realizado en las clases de prácticas para otros problemas.}
\subsection{Dasarrolle un proyecto que resuelva el problema especificado por la técnicasin hacer uso en principio de una función de cota. Dicho proyecto debe incluir un test de prueba que genere la solución para el escenario previamente descrio en el enunciado.}
\subsection{El test de prueba debe generar el archivo "GrafoAlimentos.gv" (grafo and/or relacionado de la busqueda llevada a cabo) que debe entregar incluido en el proyecto y en la memoria.}
\subsection{Modifique la solucion anterior para realizar una función de cota adecuada al problema a resolver. Ejecute de nuevo el test de prieba sobre el proyecto modificado e indique qué solución obtiene para el problema del escenario indicado previamente en el enunciado. el test de prueba debe generar el archivo "GrafoAlimentosFiltro.gv" que debe entregar incluido en el proyecto y en la memoria.}


\section{resolver el problema mediante BT, para ello:}
\subsection{Completar la ficha de descripción de la solucion mediante BT.}
\subsection{Desarrolle un proyecto que resuelva el problema especificado por técnica indicada sin hacer uso en principio de una función de cota. Dicho proyecto debe incluir un test de prieba que genere la solución para el escenario previamente descrito en el enunciado.}
\subsection{Modifique la solucion anterior para realizar una función de cota adecuada al problema a resolver. Ejecute de nuevo el test de prueba sobre el prouecto modificado e indique qué solución obtiene para el prblema del escenario indicado previamente en el enunciado.}


\section{Anexos}
\subsection{Codigo completo}
\subsection{Volcado de pantalla de los resultados obtenidos por cada prueba realizada}
\subsection{Código fuente y licencia}
Todo el código fuente del trabajo se podrá encontrar en \url{www.github.com/Andalu30/US-ADDA-PracticaIndividual2/}
a partir del lunes 4 de junio de 2018, fecha limite de entrega de este trabajo.
No se recomienda el uso de la totalidad o de parte de este trabajo en caso de que no se cambie la actividad en años posteriores debido a que la universidad usa software de detección de copias.
El contenido de este trabajo es libre y se encuentra licenciado bajo una licencia GPL v3.\\

\begin{minted}{text}
This program is free software: you can redistribute it and/or modify
it under the terms of the GNU General Public License as published by
the Free Software Foundation, either version 3 of the License, or
(at your option) any later version.

This program is distributed in the hope that it will be useful,
but WITHOUT ANY WARRANTY; without even the implied warranty of
MERCHANTABILITY or FITNESS FOR A PARTICULAR PURPOSE.  See the
GNU General Public License for more details.

You should have received a copy of the GNU General Public License
along with this program.  If not, see <http://www.gnu.org/licenses/>.

\end{minted}

\end{document}
